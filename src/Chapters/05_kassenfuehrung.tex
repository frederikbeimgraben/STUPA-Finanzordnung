% !TEX root = ../Main.tex
% ==============================================================================
% Fünfter Abschnitt: Kassenführung
% ==============================================================================

\newpage
\chapter{Kassenführung}
\sectionmark{}
\vspace*{-1.5em}
\rule{\textwidth}{0.5mm}
\vspace*{-2em}

\subsection{Zahlungsverkehr}
\begin{enumerate}[label=(\arabic*)]
\item Die Verfasste Studierendenschaft unterhält personenunabhängige Konten. Der laufende Zahlungsverkehr ist darüber abzuwickeln. Daneben ist die Unterhaltung von weiteren personenunabhängigen Anlagekonten zulässig.
\item Nur der Vorsitzende, der Finanzreferent sowie der Haushaltsbeauftragte sind für die Konten der Studierendenschaft zeichnungsberechtigt.
\item Der Zahlungsverkehr ist möglichst bargeldlos durchzuführen.
\item Rechnungen sind vom Besteller mit einem Richtigkeitsvermerk (bspw. \"Die Richtigkeit der Lieferung wird bescheinigt\") zu versehen und unverzüglich an den Haushaltsbeauftragten oder den Finanzreferenten zur Feststellung der sachlichen und rechnerischen Richtigstellung weiterzuleiten.
\item Für jede Einnahme oder Ausgabe ist ein Nachweis zu erstellen. Hierbei ist das Vier-Augen-Prinzip zu beachten.
\item Der Haushaltsbeauftragte führt in begründeten Ausnahmefällen eine Barkasse und ist bei Notwendigkeit berechtigt ggf. zusätzliche zeitlich befristete Kassen einzurichten. Das Bargeld in der Barkasse darf nicht den Betrag überschreiten, der an den nächsten fünf Tagen für die voraussichtlich zu leistenden Auszahlungen oder als Wechselgeld erforderlich ist, maximal jedoch 1.000 €.
\item Zahlungsmittel, Überweisungsaufträge und Scheckhefte sowie Sparbücher sind vom Finanzreferenten oder vom Haushaltsbeauftragten unter Verschluss zu halten.
\end{enumerate}

\subsection{Buchführung}
\begin{enumerate}[label=(\arabic*)]
\item Über alle Zahlungen ist nach der im Haushaltsplan vorgesehenen Ordnung und in zeitlicher Folge gem. den Vorgaben der LHO Buch zu führen. Die Zahlungen sind für das Haushaltsjahr zu buchen, in dem sie eingegangen oder geleistet worden sind.
\item Den Zahlungsbelegen sind die begründenden Unterlagen (wie z.B. Beschaffungsantrag, Bestellung, Vergleichsangebote, Auftragsbestätigung, Lieferschein etc.) beizulegen. Es müssen mindestens folgende Angaben enthalten sein: Bezeichnung des Titels nach dem Haushaltsplan, Datum der Auszahlung, Zahlungspartner einschl. vollständiger Adresse, Bankverbindung (außer im Falle der Abwicklung über Barkasse), Vermerk über die Feststellung der sachlichen und rechnerischen Richtigkeit, der Betrag.
\item Die Inventarverwaltung liegt in der Verantwortung des Haushaltsbeauftragten und wird entsprechend der Richtlinien der Hochschule Reutlingen zur Inventarisierung und Anlagenbuchhaltung in der jeweils aktuellen Fassung durchgeführt.
\item Belege, Kassenbücher, Kontoauszüge, Quittungsblöcke und sonstige relevante Finanzunterlagen sind nach Abschluss des Haushaltsjahres zehn Jahre lang geordnet und sicher aufzubewahren.
\end{enumerate}

\subsection{Rechnungslegung und Rechnungsprüfung}
\begin{enumerate}[label=(\arabic*)]
\item Innerhalb eines Monats nach Ende des Haushaltsjahres stellt der Finanzreferent in Abstimmung mit dem Haushaltsbeauftragten auf Grundlage der abgeschlossenen Bücher gem. §§ 81 u. 82 LHO das Rechnungsergebnis auf, welches unverzüglich dem Studierendenparlament sowie der gem. § 65 b Abs. 3 Satz 2 LHG mit der Rechnungsprüfung beauftragten Stelle/Person vorzulegen ist.
\item Die Studierendenschaft beauftragt mit deren Einvernehmen die Verwaltung der Hochschule Reutlingen mit der Rechnungsprüfung.
\item Die Mitglieder des Studierendenparlaments können jederzeit Einsicht in die Bücher verlangen.
\end{enumerate}
