% !TEX root = ../Main.tex
% ==============================================================================
% Dritter Abschnitt: Feststellung des Haushaltsplans
% ==============================================================================

\newpage
\chapter{Feststellung des Haushaltsplans}
\vspace*{-1.5em}
\rule{\textwidth}{0.5mm}
\vspace*{-2em}

\subsection{Aufstellung des Haushaltsplans}
\begin{enumerate}[label=(\arabic*)]
\item Der Entwurf des Haushaltsplans wird vom Haushaltsbeauftragten gemeinsam mit dem Finanzreferenten der Verfassten Studierendenschaft rechtzeitig vor Beginn des Haushaltsjahres aufgestellt. Der Finanzreferent legt ihn dem Studierendenparlament spätestens sechs Wochen vor Beginn des Haushaltsjahres zur Abstimmung vor.
\end{enumerate}

\subsection{Beratung im Studierendenparlament}
\begin{enumerate}[label=(\arabic*)]
\item Der Haushaltsplan wird vom Studierendenparlament per Feststellungsbeschluss verabschiedet.
\item Wird der Entwurf des Haushaltsplans vom Studierendenparlament abgelehnt, so haben der Finanzreferent und der Haushaltsbeauftragte innerhalb von 14 Tagen einen neuen Entwurf aufzustellen.
\item Der Finanzreferent hat bei der Beratung im Studierendenparlament den Entwurf des Haushaltsplans zu erläutern.
\item Der Haushaltsbeauftragte kann bei der Beratung im Studierendenparlament anwesend sein.
\end{enumerate}

\subsection{Genehmigung und Inkrafttreten des Haushaltsplans}
\begin{enumerate}[label=(\arabic*)]
\item Der vom Studierendenparlament festgestellte Haushaltsplan ist gem. § 29 Abs. 3 der Organisationssatzung spätestens einen Monat vor Beginn des Haushaltsjahres dem Präsidium der Hochschule zur Genehmigung vorzulegen. Die Vorlage und die damit zusammenhängende Kommunikation übernimmt der Haushaltsbeauftragte.
\item Der Haushaltsplan tritt mit Genehmigung durch das Präsidium zum Beginn des jeweiligen Haushaltsjahres in Kraft.
\end{enumerate}

\subsection{Nachtragshaushalt}
\begin{enumerate}[label=(\arabic*)]
\item Die Änderung eines rechtskräftigen Haushaltsplans ist nur durch einen Nachtragshaushalt möglich.
\item Auf den Nachtragshaushalt sind die Vorschriften zum Haushaltsplan sinngemäß anzuwenden, es sei denn in der Finanzordnung ist ausdrücklich etwas anderes bestimmt.
\item Ab einem Fehlbetrag von fünfzig vom Hundert der Betriebsmittelrücklage muss ein Nachtragshaushalt erstellt werden.
\end{enumerate}
