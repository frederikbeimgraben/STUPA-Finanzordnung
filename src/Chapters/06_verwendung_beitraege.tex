% !TEX root = ../Main.tex
% ==============================================================================
% Sechster Abschnitt: Verwendung der Beiträge
% ==============================================================================

\newpage
\chapter{Verwendung der Beiträge}
\sectionmark{}
\vspace*{-1.5em}
\rule{\textwidth}{0.5mm}
\vspace*{-2em}

\subsection{Verwendung der Beiträge}
Die Studierendenschaft verwendet ihre Beiträge zur Erfüllung ihrer gesetzlichen Aufgaben gem. § 65 LHG.

\subsection{Zuwendungen}
\begin{enumerate}[label=(\arabic*)]
\item Auf Antrag können studentische Initiativen, Arbeitskreise, Vereine, sonstige Einrichtungen und Funktionsträger des Studierendenparlaments zur Erfüllung der Aufgaben der Studierendenschaft gem. § 65 Abs. 2 LHG durch Zuwendungen gefördert werden. Der AStA entscheidet stellvertretend für das Studierendenparlament über die Bewilligung und die Höhe der Zuwendung.
\item Nicht zulässig sind pauschale Förderungen von Vorhaben ohne Zweckbindung, die Unterstützung von Wahlgemeinschaften sowie die Finanzierung von Aktivitäten, deren Dauer über 12 Monate hinausgeht.
\item Für Anträge, welche die Höhe der finanziellen Unterstützung im laufenden Haushaltsjahr von 5.000,- \euro (Euro) übersteigen, ist die Zustimmung der absoluten Mehrheit der anwesenden Mitglieder des Studierendenparlaments erforderlich. Das Studierendenparlament muss dazu nach §8 Abs. 1 der Satzungsordnung beschlussfähig sein.
\item Der Finanzreferent informiert die Antragsteller sowie den Haushaltsbeauftragten über die Bewilligung oder Ablehnung des Antrags.
\item Die Bewirtschaftung der bereitgestellten Mittel erfolgt durch den Haushaltsbeauftragten.
\item Restmittel werden am Ende des Haushaltsjahres den Rücklagen zugeführt.
\end{enumerate}

\subsection{Fachschaften}
\begin{enumerate}[label=(\arabic*)]
\item Das StuPa kann nach eigenem, freien Ermessen den einzelnen Fachschaften einen Etat für das Haushaltsjahr zuweisen. Über die konkrete Verwendung entscheiden die Fachschaftsvertretungen im Einvernehmen mit ihren jeweiligen studentischen Fakultätsräten. Die letztendliche Verfügung erfolgt mit Genehmigung des Finanzreferenten, welcher insbesondere die Einhaltung der haushaltsrechtlichen Grundsätze überwacht.
\item Die Bewirtschaftung der bereitgestellten Mittel erfolgt durch den Haushaltsbeauftragten.
\item Restmittel werden am Ende des Haushaltsjahres den Rücklagen zugeführt.
\end{enumerate}

\subsection{Aufwandsentschädigungen}
\begin{enumerate}[label=(\arabic*)]
\item Das Studierendenparlament kann gem. § 65 a Abs. 7 LHG i.V.m. § 33 Abs. 2 der Organisationssatzung der Studierendenschaft eine angemessene Aufwandsentschädigung festsetzen.
\item Die Höhe der Aufwandsentschädigung errechnet sich aus der Anzahl der teilgenommenen Sitzungen im Studierendenparlament mit einem dafür festgelegten Betrag pro Sitzung und wird semesterweise ausbezahlt.
\item Über die Höhe des Satzes der pro teilgenommener Sitzung ausbezahlt wird, entscheidet das Studierendenparlament und die Summe der möglichen Aufwandsentschädigungen wird im Haushaltsplan berücksichtigt.
\item Für das Haushaltsjahr wird eine Ausgabetitel für Aufwandsentschädigungen im Haushaltsplan veranschlagt. Der AStA verwaltet diesen Ausgabentitel vertretend für das Studierendenparlament und darf Aufwandsentschädigungen für außerordentliches Engagement vergeben. Diese werden zum Ende des Semesters ausgezahlt und werden zuvor im Studierendenparlament vorgestellt und beschlossen.
\end{enumerate}

\subsection{Reisekosten}
\begin{enumerate}[label=(\arabic*)]
\item Reisekosten können erstattet werden, wenn der Studierendenschaft ein nachweisbarer Nutzen aus den Reisen erwächst und sie zur Erfüllung der Aufgaben der Studierendenschaft notwendig sind.
\item Sie können nur erstattet werden, wenn die Dienstreise vorab beim AStA beantragt und durch den Vorsitzenden, sein Vertreter oder den Finanzreferenten genehmigt wurde.
\item Es gilt das Landesreisekostengesetz Baden-Württemberg nebst den dazugehörigen Vorschriften.
\end{enumerate}

\subsection{Repräsentationsausgaben}
Der Grundsatz der Wirtschaftlichkeit und Sparsamkeit erfordert vor allem bei den Bewirtungskosten besonders strenge Maßstäbe. Daher können bei rein internen Veranstaltungen hierfür keine Kosten erstattet werden. Für sonstige Veranstaltungen wird die Bewirtungsrichtlinie der Hochschule in der jeweils gültigen Fassung zugrunde gelegt.

\subsection{Beschäftigte}
\begin{enumerate}[label=(\arabic*)]
  \item Die Arbeitsverhältnisse der Beschäftigten im Dient der Studierendenschaft sind nach den für die Beschäftigten des Landes Baden-Württemberg geltenden Bestimmungen zu regeln.
  \item Einstellungen und Entlassungen von Beschäftigten werden im Rahmen der dafür im Haushaltsplan vorgesehenen Stellen und Mittel vom AStA beschlossen.
  \item Der AStA-Vorsitzende ist Dienstvorgesetzter der Beschäftigten.
  \item Eine Aufgabenänderung welche eine Änderung der Stellenwertigkeit nach sich ziehen würde, bedarf der Zustimmung von zwei Dritteln der satzungsmäßigen Mitglieder des Studierendenparlaments.
\end{enumerate}
