% !TEX root = ../Main.tex
% ==============================================================================
% Vierter Abschnitt: Ausführung des Haushaltsplans
% ==============================================================================

\newpage
\chapter{Ausführung des Haushaltsplans}
\sectionmark{}
\vspace*{-1.5em}
\rule{\textwidth}{0.5mm}
\vspace*{-2em}

\subsection{Bewirtschaftung der Einnahmen und Ausgaben}
\begin{enumerate}[label=(\arabic*)]
\item Der Haushaltsbeauftragte bewirtschaftet zusammen mit dem Finanzreferenten die Einnahmen und Ausgaben der Studierendenschaft gem. § 65 b Abs. 2 LHG i.V.m. § 9 LHO. Sie sind bei allen Maßnahmen von finanzieller Bedeutung zu beteiligen und sind für die Einhaltung der Bestimmungen dieser Ordnung sowie der haushaltsrechtlichen Vorschriften verantwortlich.
\item Ausgaben sowie Aufträge im Namen und auf Rechnung der Studierendenschaft bedürfen der vorherigen Anmeldung beim Haushaltsbeauftragten oder Finanzreferenten und erfordern dessen Zustimmung. Ausgaben oder Verpflichtungen, die 10.000 € übersteigen und nicht bereits im Haushaltsplan enthalten sind, bedürfen zusätzlich der Genehmigung des Studierendenparlaments.
\item Erhebt der Haushaltsbeauftragte Widerspruch gegen eine Maßnahme, weil er sie für rechtswidrig oder nach den Grundsätzen der Sparsamkeit und Wirtschaftlichkeit für nicht vertretbar hält, ist vom AStA-Vorsitzenden eine Entscheidung des Studierendenparlaments herbeizuführen. Bei der Entscheidung erhält der AStA ein Veto-Recht.
\item Dem Haushaltsbeauftragten und dem Finanzreferenten obliegt die Prüfung der rechnungsbegründenden Unterlagen für Lieferungen und Leistungen, die Feststellung der sachlichen und rechnerischen Richtigkeit, die Durchführung der Zahlungsvorgänge, die Verwaltung des Sachvermögens sowie die Prüfung und Abrechnung von Reisekosten.
\item Der Haushaltsbeauftragte sowie der Finanzreferent sind befugt finanzielle Angelegenheiten oder Verpflichtungen zur Aufrechterhaltung des laufenden Geschäftsbetriebes zu regeln.
\end{enumerate}

\subsection{Bevollmächtigung von Vertreterinnen und Vertretern}
\begin{enumerate}[label=(\arabic*)]
\item Der Finanzreferent kann im Rahmen einer geordneten und jederzeit übersichtlichen Wirtschaftsführung weitere Mitglieder des AStA mit der Wahrnehmung einzelner Befugnisse oder mit seiner Urlaubs- oder Krankheitsvertretung beauftragen. Hierunter fällt auch das Unterzeichnen von Anordnungen der sachlichen und rechnerischen Richtigkeit gem. § 16 Abs. 2 dieser Ordnung.
\item Die Bevollmächtigung hat schriftlich zu erfolgen und ist von der bevollmächtigten Person und von dem Vorsitzenden des AStA gegengezeichnet zu den Akten zu nehmen.
\item Die Bevollmächtigung endet unmittelbar durch schriftlichen Widerruf des Finanzreferenten, durch Ablauf einer gesetzlichen Frist, mit Ausscheiden aus dem Studierendenparlament, mit dem Ende der Amtszeit des Finanzreferenten oder durch Verlust der Geschäftsfähigkeit.
\item Der Finanzreferent hat die Handlungen der bevollmächtigten Person/en unter Berücksichtigung seiner Aufsichtspflichten angemessen zu überwachen.
\end{enumerate}

\subsection{Über- und außerplanmäßige Ausgaben}
Ausgaben, die über den Ansatz eines Titels hinausgehen oder unter keine Zweckbestimmung des Haushaltsplans fallen, können gem. § 5 der Finanzordnung getätigt werden, soweit über einen bestehenden Titel Deckungsfähigkeit besteht.

\subsection{Einhaltung des Haushaltsplans}
Alle Einnahmen und Ausgaben sind mit ihrem vollen Betrag und nur in Übereinstimmung mit der Zweckbindung der Titel einzunehmen bzw. zu verausgaben. Ist die Zuordnung zweifelhaft, so hat die Verbuchung in Gänze in einem der sich anbietenden Titel zu erfolgen. Eine Verbuchung an verschiedenen Stellen des Haushaltsplans ist in keinem Fall zulässig.

\subsection{Vorläufige Haushaltsführung}
\begin{enumerate}[label=(\arabic*)]
\item Ist der Haushaltsplan bei Beginn des Haushaltsjahres noch nicht festgestellt, darf die Studierendenschaft nur Ausgaben leisten und neue Verpflichtungen eingehen, zu denen sie rechtlich verpflichtet ist oder die für die Weiterführung der notwendigen Aufgaben unaufschiebbar sind. Ausgaben und Verpflichtungen dürfen maximal bis zur Höhe des Ansatzes des Vorjahres bzw. falls der Entwurf niedrigere Ansätze vorsieht, bis zur Höhe der Ansätze des Entwurfs geleistet bzw. eingegangen werden.
\item Neue Personalstellen sowie neue Haushaltstitel dürfen erst nach Inkrafttreten des Haushalts in Anspruch genommen werden.
\end{enumerate}

\subsection{Rücklagen}
\begin{enumerate}[label=(\arabic*)]
\item Das Studierendenparlament ist zur Unterhaltung von Rücklagen verpflichtet.
\item Das Studierendenparlament hat zur Gewährleistung einer ordnungsgemäßen Kassenwirtschaft eine Betriebsmittelrücklage zu unterhalten. Sie beträgt mindestens zehn vom Hundert und höchstens fünfzehn vom Hundert der im Haushaltsplan veranschlagten Einnahmen aus Beiträgen der Studierenden.
\item Soweit erforderlich, ist für Vermögensgegenstände von größerem Wert, die nach Alter, Verbrauch, oder aus sonstigen Gründen jeweils ersetzt werden, eine Erneuerungsrücklage, für Vermögensgegenstände, deren Bestand nach wachsendem Bedarf erweitert werden muss, sowie für besondere Vorhaben eine Erweiterungs- und Sonderrücklage anzusammeln. Die Ansammlung von Erweiterungs- und Sonderrücklagen ist erforderlich, wenn die Ausgaben aus Mitteln des Haushalts voraussichtlich nicht bestritten werden können.
\item Die Rücklagen sowie andere nicht sofort benötigte Finanzmittel sind bei mündelsicheren Kreditinstituten als Spareinlagen oder Termingelder ohne die Möglichkeit des Verlustes, nach Bedarf verfügbar in Euro anzulegen.
\item Zinsen aus Rücklagebeständen sind im Haushaltsplan zu veranschlagen. Sie fließen nicht den Rücklagen zu, sondern sind als Einnahmen nachzuweisen.
\end{enumerate}
