% !TEX root = ../Main.tex
% ==============================================================================
% Zweiter Abschnitt: Haushaltsplan
% ==============================================================================

\newpage
\chapter{Haushaltsplan}
\vspace*{-1.5em}
\rule{\textwidth}{0.5mm}
\vspace*{-2em}

\subsection{Grundsätze}
\begin{enumerate}[label=(\arabic*)]
\item Der Haushaltsplan und etwaige Nachträge werden unter Berücksichtigung des zur Erfüllung der Aufgaben notwendigen Bedarfs federführend durch den Finanzreferenten und gem. § 65b Abs. 2 LHG i.V.m. § 9 LHO dem Haushaltsbeauftragten entworfen und vom Studierendenparlament durch Beschluss festgestellt. Er bildet die Grundlage der Verwaltung aller Einnahmen, Ausgaben und Verpflichtungsermächtigungen für die Buchführung und Rechnungslegung.
\item Einnahmen und Ausgaben sind getrennt voneinander in voller Höhe zu veranschlagen.
\item Für den gleichen Einzelzweck dürfen Mittel nicht an verschiedenen Stellen des Haushaltsplans veranschlagt werden.
\item Der Haushaltsplan hat in Einnahmen und Ausgaben ausgeglichen zu sein.
\item Bei der Aufstellung und Ausführung des Haushaltsplans sind die Grundsätze der Wirtschaftlichkeit und Sparsamkeit zu beachten.
\end{enumerate}

\subsection{Haushaltsjahr}
\begin{enumerate}[label=(\arabic*)]
\item Das erste Haushaltsjahr begann am 1. Dezember 2013 und endete am 28. Februar 2015.
\item Alle nachfolgenden Haushaltsjahre beginnen am 1. März mit dem Anfang des Sommersemesters und enden am 28. Februar mit dem Ende des Wintersemesters des folgenden Kalenderjahres.
\end{enumerate}

\subsection{Veranschlagung der Einnahmen, Ausgaben und Stellen}
\begin{enumerate}[label=(\arabic*)]
\item Der Haushaltsplan besteht aus Einnahme- und Ausgabetiteln mit jeweils fester Zweckbestimmung. Die Einnahmen sind nach dem Entstehungsgrund, die Ausgaben nach Zwecken getrennt den Titeln zuzuordnen und soweit erforderlich, zu erläutern. Die Zuordnung ist so vorzunehmen, dass aus dem Haushaltsplan die Erfüllung der Aufgaben der Studierendenschaft erkennbar ist. Im Haushaltsplan sind mindestens gesondert darzustellen: 1. bei den Einnahmen: Studierendenschaftsbeiträge, Einnahmen aus wirtschaftlicher Betätigung, Einnahmen aus nicht wirtschaftlichen Betätigungen, Entnahmen aus Rücklagen und sonstige Einnahmen. 2. bei den Ausgaben: Personalausgaben, sachliche Verwaltungsausgaben, Zuwendungen an Stellen außerhalb der Studierendenschaft, Ausgaben für wirtschaftliche Betätigung, Investitionen und Zuführungen an Rücklagen. Im Haushaltsplan kann bestimmt werden, dass Mehr- oder Mindereinnahmen, die in sachlichem Zusammenhang mit bestimmten Ausgaben stehen, die betreffenden Ausgabensätze erhöhen oder verringern. Voraussichtlich benötigte Verpflichtungsermächtigungen sind bei den jeweiligen Ausgaben gesondert zu veranschlagen. Im Haushaltsplan können Ausgaben und Verpflichtungsermächtigungen jeweils für gegenseitig oder einseitig deckungsfähig erklärt werden, wenn ein verwaltungsmäßiger oder sachlicher Zusammenhang besteht oder eine wirtschaftliche und sparsame Verwendung gefördert wird.
\item Die Titel sind mit einem Ansatz (Betrag) auszubringen. Die Ansätze sind in ihrer voraussichtlichen Höhe zu errechnen oder soweit dies nicht aufgrund von Unterlagen möglich ist, sorgfältig zu schätzen.
\item Neben dem Ansatz für das aktuelle Haushaltsjahr sind auch der Ansatz sowie das Rechnungsresultat des Vorjahres in den Haushaltsplan aufzunehmen.
\item Dem Haushaltsplan werden folgende Anlagen hinzugefügt: 1. Übersicht über die Rücklagen (Vermögensübersicht) 2. Darstellung der Einnahmen, Ausgaben und Verpflichtungsermächtigungen 3. Übersicht über die den Haushalt in Einnahmen und Ausgaben durchlaufenden Posten 4. Übersicht über die Planstellen und die anderen Stellen 5. Übersicht über das Vermögen und die Schulden.
\end{enumerate}

\subsection{Gegenseitig deckungsfähige Titel}
Die einzelnen Titel des Haushaltsplans sind gegenseitig deckungsfähig, solange im Haushaltsplan nichts Abweichendes vermerkt ist.

\subsection{Bedeutung des Haushaltsplans gegenüber Dritten}
Durch den Haushaltsplan werden Ansprüche oder Verbindlichkeiten Dritter weder begründet noch aufgehoben.
